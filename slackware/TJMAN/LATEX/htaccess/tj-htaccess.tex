%%%%%%%%%%%%%%%%%%%%%%%%%%%%%%%%%%%%%%%%%%%%%%%%%%%%%%%
% (c) technojuice 2003
% 
% LaTeX source sample for internal docs
% 
% skills:
%         - landscape layout
%         - chapter pers
%         - output: pdf & HTML
%
%%%%%%%%%%%%%%%%%%%%%%%%%%%%%%%%%%%%%%%%%%%%%%%%%%%%%%%

%\documentclass[a4paper,12pt,landscape]{report} 	% landscape for code sample
% twoside,openright per fronte retro (recto-verso)default per book
\documentclass[a4paper,12pt,twoside,titlepage]{report} 	

\voffset=-70pt
\textwidth=410pt		% width page
\textheight=650pt  	% height page
\headsep=24pt
%\footskip=32pt
\evensidemargin=30pt
\oddsidemargin=30pt

\usepackage[T1]{fontenc}
\usepackage[latin1]{inputenc}										%	ISO-8859-1 for �� etc..
\usepackage[italian]{babel}
\usepackage[Lenny]{fncychap}									% fncychap from U. Lindgren best formatting chapeters, Lenny style
\usepackage{graphicx}													% to include img EPS
\usepackage{makeidx}													% to create anlit index

\title{Configure Htaccess }
\author{guido@technojuice.net}
\date{5 marzo 2001}
\makeindex
\begin{document}
%\begin{titlepage}
\maketitle
%\begin{figure}[htbp]		% env figure
%\begin{center}					% env center
%\includegraphics [width=2cm]{cables.eps}
%\end{center}
%\end{figure}
%\end{titlepage}




\tableofcontents
\chapter{Configurazione di htaccess}
\section{Direttive httpd}
\begin{itemize}
\item configurazione APACHE, in /etc/httpd/httpd.conf\newline abilitare nella sezione
 <DIRECTORY> la direttiva: AllowOverride AuthConfig di default impostata su 'none'.

\item creazione file delle password:\newline 
da /usr/bin chiamare il binario htpasswd; la prima volta che viene utilizzato occorre il parametro -c  
\emph{'\index{htpasswd}~-c /etc/httpd/newpasswordfile username'}, dove il path corrisponde alla destinazione del file delle password chiamato per default 'htpasswd' ma potrebbe anche essere un file nascosto, username � il primo utente creato, ripetere per ogni utente la procedura senza il\newline parametro~-c. La modifica \index{password} viene effettuata invocando htpasswd nuovo 'username' passwordfile creato in precedenza. 

\item creazione della directory con accesso autorizzato: 
semplicemente aggiungendo un file nascosto \index{.htaccess} alla directory che si intende proteggere. 
\end{itemize}

% fin qui ok
\section{Direttive .htaccess}
\begin{description}
\item[Options]Opzioni varie 
\item[ErrorDocument]Personalizza la risposta alle condizioni di errore 
\item[DefaultType]Definisce il mime predefinito per la directory 
\item[AuthName]Nome o descrizione di una zona soggetta ad autenticazione
\item[AuthType]Definizione del tipo di autenticazione (Basic)
\item[require]Dichiarazione degli utenti autorizzati all'autenticazione 
\item[order]Ordine di valutazione delle direttive deny e allow 
\item[deny]Specifica gli host a cui viene negato l'accesso 
\item[allow]Specifica gli host a cui viene concesso l'accesso 
\end{description}

\section{files .htaccess}
\subsection{Valid users based}
\begin{center}
\begin{verbatim}
AuthUserFile /etc/httpd/users
	AuthName "commento"
	AuthType Basic
	<Limit GET>
	require valid-user
	</Limit>
\end{verbatim}
\end{center}
autorizza TUTTI gli utenti iscritti nel files 'users' 

\subsection{Valid Group}
\begin{center}
\begin{verbatim}
AuthUserFile /etc/httpd/users
AuthGroupFile /etc/httpd/htgroup
AuthName ByPassword
AuthType Basic
<Limit GET>
require group my-users
</Limit> 
\end{verbatim}
\end{center}
Per facilitare la gestione di utenti che accedono a directory protette, possiamo realizzare dei gruppi e inserirli in un file chiamato .htgroup, il quale non contiene alcuna password. La sintassi dei record che compongono questo file � molto semplice, infatti raggruppa alcuni utenti, i cui nominativi e password sono specificati nel file delle password 'users'. Sintassi: nomegruppo: utente1 utente2 utenteN 

\subsection{Valid user}
\begin{center}
\begin{verbatim}
AuthUserFile /etc/httpd/users
AuthName "commento"
AuthType Basic
<Limit GET>
require user pippo
\end{verbatim}
\end{center}
Solo l'utente 'pippo' potr� accedere alla directory. 

\section{note} 
le direttive 'options' in .htaccess, possono avere la precedenza sulle impostazioni globali date da \index{httpd.conf}, conseguentemente sono convenienti per abilitare localmente alcune impostazioni soprattutto in mancanza della possibilit� di modificare httpd.conf 

\printindex
\end{document}